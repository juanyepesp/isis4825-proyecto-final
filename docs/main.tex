\documentclass[conference]{IEEEtran}
\usepackage[utf8]{inputenc}
\usepackage{graphicx}
\usepackage{amsmath}
\usepackage{url}
\usepackage[spanish]{babel}

\title{Aproximación a la identificación de la minería Ilegal en videos de cámaras FLIR a partir de la detección y la diferenciación de carreteras no pavimentadas y cuerpos de agua}

\author{
    \IEEEauthorblockN{
        María Alejandra Ariza Rangel\IEEEauthorrefmark{1}, 
        Camilo Andrés Daza Ramírez\IEEEauthorrefmark{2},\\
        María Paola Reyes Gómez\IEEEauthorrefmark{1}, 
        Juan Diego Yepes Parra\IEEEauthorrefmark{3}
    }
    \IEEEauthorblockA{
        \IEEEauthorrefmark{1}Maestría en Biología Computacional,\\
        \IEEEauthorrefmark{2}Pregrado en Ingeniería de Sistemas y Computación,\\
        \IEEEauthorrefmark{3}Maestría en Ingeniería de Sistemas y Computación,\\
        Facultad de Ingeniería, Universidad de los Andes, Bogotá, Colombia
    }
    
}


\begin{document}

\maketitle

\begin{abstract}
    La Fuerza Aeroespacial Colombiana (FAC), en alianza con la Universidad de los Andes, busca modernizar sus procesos mediante soluciones innovadoras para la protección ambiental. Su principal objetivo es preservar la Amazonia Colombiana mediante el análisis de videos capturados con sensores aerotransportados para detectar minería ilegal.

    Este análisis, tradicionalmente manual, ahora se enfocará en la detección automatizada de carreteras no pavimentadas y cuerpos de agua. Esta automatización optimiza la identificación de áreas afectadas, mejorando la precisión y eficiencia del proceso.

    Se prevén tres beneficios clave: reducción del tiempo y errores en la identificación mediante inteligencia artificial, mejora en la segmentación de imágenes y fortalecimiento de la toma de decisiones en operaciones militares. 

    Además, esta tecnología tiene aplicaciones más amplias, como el apoyo a imágenes satelitales para la planeación militar, el monitoreo de la deforestación y la gestión de recursos hídricos. También puede contribuir a la formulación de políticas ambientales, al fortalecimiento de la seguridad en la región y al desarrollo de nuevas investigaciones.    
\end{abstract}

\begin{IEEEkeywords}
Palabras clave; separadas; por punto y coma.
\end{IEEEkeywords}

\section{Introducción}
Contenido de la introducción \cite{exampleReference}.

\section{Estado del Arte}
Contenido del estado del arte.

\section{Materiales}
Descripción de los materiales utilizados.

\section{Métodos}
Metodología empleada.

\section{Resultados}
Presentación de los resultados obtenidos.

\section{Discusión y Conclusión}
Análisis, discusión y conclusiones.

\bibliographystyle{IEEEtran}
\bibliography{references}

\end{document}
