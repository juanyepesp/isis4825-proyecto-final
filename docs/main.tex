\documentclass[conference]{IEEEtran}
\usepackage[utf8]{inputenc}
\usepackage{graphicx}
\usepackage{amsmath}
\usepackage{hyperref}
\usepackage{url}
\usepackage[spanish]{babel}

\title{Aproximación a la identificación de la minería Ilegal en videos de cámaras FLIR a partir de la detección y la diferenciación de carreteras no pavimentadas y cuerpos de agua}

\author{
    \IEEEauthorblockN{
        María Alejandra Ariza Rangel\IEEEauthorrefmark{1}, 
        Camilo Andrés Daza Ramírez\IEEEauthorrefmark{2},\\
        María Paola Reyes Gómez\IEEEauthorrefmark{1}, 
        Juan Diego Yepes Parra\IEEEauthorrefmark{3}
    }
    \IEEEauthorblockA{
        \IEEEauthorrefmark{1}Maestría en Biología Computacional,\\
        \IEEEauthorrefmark{2}Pregrado en Ingeniería de Sistemas y Computación,\\
        \IEEEauthorrefmark{3}Maestría en Ingeniería de Sistemas y Computación,\\
        Facultad de Ingeniería, Universidad de los Andes, Bogotá, Colombia,\\
        Facultad de Ciencias, Universidad de los Andes, Bogotá, Colombia
    }
    
}


\begin{document}

\maketitle

\begin{abstract}
    % TODO
    La Fuerza Aeroespacial Colombiana (FAC), en colaboración con la Universidad de los Andes, desarrolló un sistema automatizado para detectar minería ilegal en la Amazonía mediante el análisis de imágenes térmicas FLIR capturadas por sensores aerotransportados. El sistema se centra en la detección de carreteras no pavimentadas y cuerpos de agua como indicadores indirectos de actividad minera.
    
    Se propuso un esquema de preprocesamiento multicanal que mejora la calidad visual de las imágenes térmicas al combinar representaciones sintéticas optimizadas para texturas, estructuras y segmentación espacial. Posteriormente, se evaluaron modelos de aprendizaje profundo, destacándose YOLOv11l con un mAP@50 de 65.2\% y YOLOv11s con un mAP@50–95 de 32.8\%, superando incluso a modelos más robustos. En contraste, RetinaNet obtuvo resultados limitados (mAP@50–95 de 12.2\%) debido a su sensibilidad al desbalance de clases y a las distorsiones térmicas.
    
    Los resultados validan la viabilidad técnica del enfoque propuesto y su potencial integración en sistemas reales de monitoreo ambiental y operaciones militares. El sistema ofrece beneficios clave como la reducción del tiempo de análisis, la mejora en la segmentación y la toma de decisiones más precisa. Además, puede escalarse a otras aplicaciones, como la detección de deforestación, la gestión de recursos hídricos y el soporte a políticas públicas de conservación ambiental.    
\end{abstract}

\begin{IEEEkeywords}
Palabras clave; separadas; por punto y coma. %TODO
\end{IEEEkeywords}

\section{Introducción}

\subsection{Objetivo General}

Diseñar y desarrollar un modelo basado en aprendizaje profundo (deep learning) para el procesamiento de imágenes extraídas de videos, para mejorar su calidad, eliminar ruido y automatizar la detección de cuerpos de agua y carreteras no pavimentadas, facilitando así la identificación automática de estos elementos y contribuyendo al reconocimiento de campamentos de minería ilegal.

\subsection{Objetivos Específicos}

\begin{enumerate}
    \item Implementar técnicas de preprocesamiento de imágenes que permitan mejorar la calidad de los fotogramas extraídos de los videos, eliminando ruido y mejorando el contraste, con una reducción mínima del 20\% en la distorsión de la imagen. 
    \item Diseñar y entrenar un conjunto de datos etiquetado para la identificación de carreteras no pavimentadas y cuerpos de agua.
    \item Desarrollar un modelo de aprendizaje profundo basado en redes neuronales convolucionales para la detección y clasificación de carreteras no pavimentadas y cuerpos de agua en videos capturados por cámaras FLIR.
    \item Evaluar el desempeño del modelo mediante métricas de clasificación como precisión, recall y F1-score, con un umbral mínimo de 80\% en cada una, para validar su aplicabilidad en escenarios reales de detección de minería ilegal.
\end{enumerate}
\section{Estado del arte}

Recientemente, el aprendizaje automático y la visión por computadora han mejorado significativamente el monitoreo ambiental mediante imágenes satelitales, especialmente en áreas como la detección de pistas clandestinas en la Amazonía. En este contexto, Pardini et al. \cite{pardini2025} desarrollaron un modelo basado en redes neuronales para reducir los falsos positivos en un 26.6\%, optimizando el procesamiento de 43 a 32 horas mediante paralelización.

En la gestión hídrica, Albuquerque Teixeira et al. \cite{albuquerque2024} aplicaron redes neuronales para segmentar embalses en Brasil, alcanzando una precisión del 95\% en la Intersección sobre Unión (IoU), mejorando así la monitorización de recursos hídricos en zonas semiáridas.

Por otro lado, Ferreira et al. \cite{ferreira2019} implementaron una fusión de datos de teledetección para detectar cultivos ilícitos, alcanzando una precisión del 92.16\% y reduciendo los falsos positivos al 5.87\%. Este enfoque es particularmente útil para identificar cultivos ilegales, como la \textit{Cannabis Sativa}, en Brasil.

Pinto Hidalgo et al. \cite{pinto2023} crearon un modelo con redes neuronales y datos geoespaciales para detectar infraestructuras de producción de coca en la frontera Venezuela-Colombia. Su metodología mejoró la vigilancia en zonas de difícil acceso, usando imágenes satelitales y bases de datos geoespaciales.

En cuanto a la detección de carreteras ilegales, Sloan et al. \cite{sloan2024} utilizaron redes U-Net y ResNet-34 para identificar carreteras en Asia Pacífico con una precisión entre el 72\% y el 81\%. Su metodología podría aplicarse en ecosistemas tropicales para monitorear el impacto ambiental de infraestructuras no autorizadas.

A pesar de estos avances, aún existen desafíos en la optimización de modelos para diferentes condiciones geográficas y climáticas, y en la reducción de costos computacionales sin afectar la precisión. Estos estudios ofrecen valiosas herramientas para mejorar la vigilancia ambiental y la gestión sostenible de los recursos naturales.

\section{Métodos}
\subsection{Obtención de datos}
Se dispone de 25 minutos de video etiquetado, el cual ha sido dividido en alrededor de 4,000 fotogramas en formato PNG. Estos fotogramas pertenecen a las Fuerzas Aéreas Colombianas (FAC) y provienen de cámaras FLIR (Forward-Looking Infrared), sensores térmicos avanzados que permiten la detección de objetos y actividades a partir de la radiación infrarroja emitida por los cuerpos. Los fotogramas incluyen diversas capturas de territorios con características variadas, tales como montañas, cuerpos de agua, carreteras, vehículos, entre otros. Además, se cuenta con las coordenadas geográficas de las imágenes para facilitar la identificación de objetos dentro de ellas. No obstante, los fotogramas presentan ciertos problemas, como exceso de ruido, distorsión, sombras y baja resolución; así que fue necesario hacer un preprocesamiento de éstas (ver sección \ref{prepro_datos}). 

\subsection{Preprocesamiento de las imágenes} \label{prepro_datos}

Los datos fueron preprocesados en Python utilizando la biblioteca \texttt{scikit-image}, la cual forma parte del ecosistema de \texttt{scikit-learn} y ofrece una amplia colección de algoritmos para el procesamiento de imágenes. Las imágenes empleadas contienen tres canales correspondientes al modelo RGB (rojo, verde y azul). Para el preprocesamiento, se trabajó inicialmente con cada banda por separado y, posteriormente, se integraron para formar la imagen compuesta. 

Adicionalmente, se aplicaron técnicas de aumentación de datos con el objetivo de incrementar el número de muestras, aportar mayor variabilidad y reducir el riesgo de sobreajuste (\textit{overfitting}). Esta fase se llevó a cabo manipulando las representaciones de cada imagen a través de sus respectivos canales.

En el \textbf{canal rojo}, el preprocesamiento se enfocó en la reducción de ruido y la mejora de la calidad de la imagen. Inicialmente, se aplicó un filtro gaussiano, que suaviza la imagen sin introducir artefactos indeseados mediante una distribución gaussiana que pondera los píxeles vecinos. Se utilizó un valor de \texttt{sigma} = 1.5, correspondiente a un suavizado intermedio. La imagen resultante se convirtió al tipo de datos \texttt{uint8}. Luego, se aplicó un filtro mediano, útil para eliminar ruido del tipo sal y pimienta y preservar los bordes. Este filtro se aplicó con una ventana de 5x5 píxeles.

Finalmente, se aplicó un filtro guiado no lineal que emplea la imagen original como guía para suavizar la imagen previamente filtrada. Este tipo de filtro conserva los bordes con mayor precisión. Se utilizó un radio de 5 y un valor de $\epsilon = 10^{-2}$, lo suficientemente bajo para preservar detalles importantes. El resultado fue una imagen suavizada con bordes claramente definidos, lo cual es fundamental para identificar adecuadamente elementos como carreteras sin pavimentar y cuerpos de agua.

En el \textbf{canal azul}, se utilizó la segmentación basada en el modelo de \textit{Chan-Vese}, que separa automáticamente la imagen en regiones homogéneas en función de similitudes en color, textura o intensidad. El objetivo fue distinguir entre la \textit{región de interés} (carreteras y cuerpos de agua) y el fondo. 

Los parámetros empleados fueron:

\begin{itemize}
    \item $\mu = 0.25$: regula el suavizado de las fronteras segmentadas. Un valor bajo favorece la preservación de bordes.
    \item \texttt{tol} $= 10^{-3}$: umbral de tolerancia para la convergencia del algoritmo.
    \item Número máximo de iteraciones: 100, con un paso de actualización de 0.5, lo cual promueve una segmentación precisa.
\end{itemize}

Este proceso permitió aislar regiones homogéneas, facilitando la eliminación del fondo y resaltando los objetos de interés.

En la \textbf{banda verde} se aplicó un filtro lineal del tipo Gabor, ideal para capturar patrones espaciales, especialmente texturas, en diferentes orientaciones y escalas. Se construyó un banco de filtros Gabor con diversas orientaciones, cuyas respuestas se promediaron para combinar la información de textura. Posteriormente, la imagen combinada fue normalizada al rango [0, 255], preparando los valores de los píxeles para su incorporación en el canal verde.

Este procesamiento es clave para resaltar patrones y texturas útiles en tareas de segmentación y detección de objetos en escenas complejas.

Finalmente, las imágenes resultantes de cada canal fueron combinadas para formar una única imagen RGB en la que cada banda aporta información única y complementaria.

\subsection{Modelos de Detección de Objetos}

Para llevar a cabo la detección de objetos en las imágenes, se implementaron diversos modelos de aprendizaje profundo, específicamente: \textbf{YOLO}, \textbf{Faster R-CNN} y \textbf{RetinaNet}.


\subsubsection{YOLO (You Only Look Once)}
YOLO es un algoritmo de detección de objetos en tiempo real ampliamente utilizado en visión por computador. Su enfoque consiste en dividir una imagen en una cuadrícula y, en una sola pasada, predecir simultáneamente las cajas delimitadoras (bounding boxes) y las clases de los objetos. Gracias a esta arquitectura unificada, YOLO ofrece un excelente equilibrio entre velocidad y precisión.

En este proyecto se trabajó con tres variantes de YOLO, todas entrenadas con imágenes propias del conjunto de datos.

\begin{itemize}
    \item \textbf{YOLO Small:} 
    Esta versión del modelo es más liviana y está optimizada para funcionar en dispositivos con recursos computacionales limitados. Se entrenó desde cero utilizando la implementación de la librería \textit{Ultralytics}. El proceso de entrenamiento se realizó con un máximo de \textbf{50 épocas} y un tamaño de imagen de \textbf{640 píxeles}, buscando un equilibrio entre velocidad de entrenamiento y precisión. Se utilizó un \textbf{batch size de 16}, y se aplicó la técnica de \textit{early stopping} con una paciencia de 8 épocas. Además, se habilitó la \textbf{aumentación de datos} para mejorar la capacidad de generalización del modelo, introduciendo transformaciones aleatorias como rotaciones, escalados y variaciones en la iluminación. Se registró el tiempo total de entrenamiento para evaluar la eficiencia del modelo.

    \item \textbf{YOLO Large:}
    Esta variante representa una versión más robusta y precisa del modelo YOLO. También fue entrenada desde cero con el dataset propio, siguiendo una configuración similar a la del modelo Small: \textbf{50 épocas}, tamaño de imagen de \textbf{640 píxeles}, \textbf{batch size de 16} y \textbf{early stopping} con paciencia de 8 épocas. La diferencia principal radica en su mayor capacidad de representación, lo que se traduce en un mejor rendimiento en tareas complejas, aunque con un mayor consumo de memoria y tiempo de entrenamiento. Se activó igualmente la aumentación de datos para reforzar la generalización del modelo ante condiciones variadas.

    \item \textbf{YOLO Large con fine-tuning (congelando el backbone):}
    En esta tercera configuración se aplicó una estrategia de \textit{transfer learning}, partiendo de un modelo YOLO Large preentrenado en un conjunto de datos extenso. Se congeló el \textbf{backbone} (capas encargadas de extraer características generales), entrenando únicamente las capas finales responsables de la detección (\textit{head}). Para ello, se bloquearon explícitamente las primeras 14 capas del modelo (\texttt{model.0} a \texttt{model.13}), lo cual se reforzó utilizando el parámetro \texttt{freeze=14}. Como resultado, solo alrededor del \textbf{3\%} de los parámetros del modelo fueron entrenables, reduciendo significativamente el tiempo y los recursos necesarios. Esta variante fue entrenada durante \textbf{20 épocas}, con un \textbf{batch size de 8}, utilizando un \textbf{learning rate reducido de 0.001} y el optimizador \textbf{Adam}, ideal para ajustes rápidos. Se habilitó el uso de GPU si estaba disponible, y el modelo resultante fue exportado en formato \textbf{ONNX}, permitiendo su integración en sistemas de inferencia en tiempo real.
    Esta estrategia permitió adaptar eficazmente el modelo a un nuevo dominio de aplicación, aprovechando los conocimientos previos del modelo base.
\end{itemize}

\subsubsection{Faster R-CNN}  
Faster R-CNN es un modelo de detección de objetos de dos etapas ampliamente reconocido por su alta precisión. Su arquitectura combina una red de propuestas de regiones (\textit{Region Proposal Network}, RPN) con una segunda etapa que clasifica y ajusta las cajas generadas. En este trabajo, se entrena el modelo desde cero utilizando imágenes propias y anotaciones personalizadas. El pipeline de entrenamiento y evaluación fue desarrollado en PyTorch, partiendo de un conjunto de imágenes anotadas en formato YOLO, cuyas etiquetas son convertidas a coordenadas absolutas \([x_1, y_1, x_2, y_2]\) compatibles con Faster R-CNN. El modelo base es una versión preentrenada de \texttt{fasterrcnn\_resnet50\_fpn}, al cual se le reemplaza la cabeza de clasificación para detectar dos clases específicas: ``carretera'' y ``río'', además de la clase de fondo. Durante el entrenamiento, se emplean técnicas como \textit{gradient clipping} para mejorar la estabilidad numérica, \textit{learning rate scheduling} para una mejor convergencia, y \textit{early stopping} para prevenir el sobreajuste. El mejor modelo se guarda automáticamente según su rendimiento en la validación. Aunque este enfoque conlleva mayores tiempos de inferencia comparado con modelos más livianos como YOLO, ofrece resultados sobresalientes en términos de precisión.

\subsubsection{RetinaNet}  
RetinaNet es un modelo de detección de objetos de una sola etapa que destaca por su uso de la función de pérdida \textit{Focal Loss}, diseñada específicamente para mitigar el impacto del desbalance de clases durante el entrenamiento. El modelo fue entrenado desde cero utilizando nuestras propias imágenes, implementando un pipeline completo en PyTorch y Torchvision. Se define una clase personalizada \texttt{RetinaNetDataset} que carga imágenes y etiquetas en formato YOLO, transformando las coordenadas normalizadas en cajas delimitadoras absolutas compatibles con RetinaNet. Se construyen \textit{DataLoaders} para los conjuntos de entrenamiento y validación, garantizando un rendimiento eficiente. Durante el entrenamiento, se permite la actualización de todos los parámetros del modelo para maximizar el aprendizaje.


Para todos los modelos, se evalúa el rendimiento periódicamente utilizando métricas estándar como \textit{mAP@0.5} y \textit{mAP@0.5:0.95}, que miden la precisión en la detección y clasificación correcta de objetos. Este enfoque asegura una evaluación rigurosa y comparativa de la efectividad de los modelos entrenados.

\subsection{Despliegue}

Para facilitar la interacción del usuario con los modelos de detección de ríos y carreteras, se diseñó un sistema modular compuesto por dos componentes principales: un \textbf{frontend} y un \textbf{backend}. Esta arquitectura se eligió con el objetivo de separar las responsabilidades entre la presentación de la interfaz y la lógica de procesamiento, permitiendo así una mayor flexibilidad en el desarrollo.

El \textbf{backend} fue planteado en \texttt{Python}, dado que todos los modelos fueron desarrollados y entrenados en este lenguaje utilizando bibliotecas como \texttt{PyTorch} y \texttt{Ultralytics}. Se propuso el uso de un framework ligero como \texttt{FastAPI} para exponer servicios web que reciban imágenes desde el cliente, procesen la solicitud con el modelo seleccionado por el usuario y retornen los resultados en un formato estructurado. La arquitectura considera la posibilidad de cargar dinámicamente distintos modelos según la petición, evitando mantenerlos todos en memoria de forma simultánea.

El \textbf{frontend} se desarrolló utilizando el framework \texttt{Svelte}, debido a su simplicidad, bajo peso y rapidez de desarrollo. La interfaz fue concebida como una herramienta sencilla, enfocada en la usabilidad: permite al usuario cargar una o varias imágenes, seleccionar el modelo de detección que desea emplear, y enviar la solicitud al backend. La comunicación entre frontend y backend se plantea mediante peticiones HTTP tipo POST con los archivos y parámetros necesarios.

En cuanto al despliegue, se contempló inicialmente un entorno local para facilitar la prueba y validación del sistema durante el desarrollo. No obstante, dado el requerimiento de que este despliegue debe ser consultable para el equipo docente, se utilizaron contenedores (\texttt{Docker}) con servicios de la nube de AWS, para poder tener la solución en un ambiente productivo.

Todo el código fuente del proyecto, el cual contiene esta misma documentación, el back-end, front-end, y los notebooks utilizados para varios propósitos, se encuentran en el siguiente repositorio público de Github: \href{https://github.com/juanyepesp/isis4825-proyecto-final}{https://github.com/juanyepesp/isis4825-proyecto-final}
\section{Resultados}

% TODO
\section{Discusión y Conclusión}

El desarrollo de un sistema automatizado de detección de minería ilegal basado en imágenes térmicas representa un avance crucial en la transformación digital de las estrategias de vigilancia ambiental y defensa territorial. La combinación entre el preprocesamiento multicanal y el uso de modelos de aprendizaje profundo permitió superar limitaciones significativas en el análisis de imágenes FLIR, tales como el ruido térmico, la baja resolución espacial, y la interferencia atmosférica inherente a la región amazónica colombiana.

El esquema de preprocesamiento propuesto —basado en una representación RGB sintética donde cada canal fue optimizado para capturar distintas características visuales: suavizado estructural (R), texturas (G) y segmentación espacial (B)— se consolidó como un componente clave para enriquecer la representación de las imágenes y, por ende, potenciar el entrenamiento de los modelos. Esta estrategia fue coherente con enfoques previos en teledetección, como los descritos por Ferreira et al. (2019), quienes combinaron múltiples fuentes de datos remotos para mejorar la identificación de cultivos ilícitos en Brasil, alcanzando una precisión del 92.16 \%
. Asimismo, Teixeira et al. (2024) lograron una segmentación precisa de embalses mediante redes neuronales profundas y preprocesamiento específico, alcanzando una precisión de 95 \% en la métrica IoU
.

Entre los modelos evaluados, YOLO11l demostró el mejor desempeño con un mAP@50 de 65.2 \%, confirmando su capacidad para detectar con alta certeza elementos relevantes en imágenes complejas. Es destacable también el rendimiento del modelo YOLO11s, que con tan solo 9.4 millones de parámetros logró un mAP@50–95 de 32.8 \%, superando incluso a su contraparte más robusta. Este hallazgo sugiere que modelos más livianos pueden ser más eficientes en contextos donde la localización precisa es más crítica que la clasificación general. Estas observaciones son coherentes con la literatura, en la cual modelos YOLO optimizados han demostrado ser eficientes y precisos en entornos con restricciones de hardware y condiciones de iluminación adversas (Sloan et al., 2024)
.

En contraste, RetinaNet presentó un desempeño deficiente (mAP@50–95 de 12.2 \%), posiblemente debido a su mayor sensibilidad al desbalance de clases y a las distorsiones térmicas propias de los sensores FLIR. Esta susceptibilidad también fue observada por Sloan et al. (2024), quienes destacaron la necesidad de adaptar arquitecturas específicas para condiciones atmosféricas y geomorfológicas extremas
.

Un aspecto limitante fue la naturaleza indirecta de las clases detectadas. Aunque el modelo logró identificar carreteras no pavimentadas y cuerpos de agua —ambos elementos comúnmente asociados con la minería ilegal—, la ausencia de clases explícitas como dragas, campamentos o balsas restringe su aplicabilidad directa para pruebas legales o intervenciones operativas. Pinto Hidalgo et al. (2023) subrayan la importancia de combinar imágenes satelitales con datos geoespaciales para mejorar la precisión en la detección de infraestructuras ilegales de producción de coca, lo que sugiere una posible línea de mejora para este proyecto
.

Desde una perspectiva operativa, los resultados alcanzados son altamente alentadores. La integración de estos modelos en sistemas de vigilancia aérea de la Fuerza Aeroespacial Colombiana (FAC) podría reducir significativamente los tiempos de análisis y aumentar la precisión en la identificación temprana de amenazas ambientales. Además, el sistema podría escalarse para apoyar tareas de monitoreo de deforestación, planificación territorial y gestión de recursos hídricos, en línea con aplicaciones discutidas por Teixeira et al. (2024) y Ferreira et al. (2019)
.

En términos éticos y sociales, es imperativo asegurar que estas tecnologías se apliquen con protocolos de supervisión humana, salvaguardando los derechos de las comunidades locales. La inteligencia artificial no puede ni debe reemplazar el juicio humano, especialmente cuando se trata de intervenciones en territorios habitados o culturalmente sensibles.

Finalmente, este trabajo abre múltiples posibilidades de investigación. En particular, el uso de modelos multimodales que combinen imágenes térmicas con datos ópticos o espectrales, así como la implementación de arquitecturas basadas en transformers o aprendizaje auto-supervisado, podría mejorar la adaptabilidad del sistema a nuevos dominios con escasa disponibilidad de datos etiquetados.

VI. Conclusiones
Este proyecto demuestra la viabilidad técnica y operativa de utilizar algoritmos de aprendizaje profundo, especialmente modelos YOLO y Faster R-CNN, para automatizar la detección de carreteras no pavimentadas y cuerpos de agua en imágenes térmicas FLIR capturadas por plataformas aéreas de vigilancia. La implementación de un esquema de preprocesamiento multicanal permitió mejorar notablemente la calidad de las imágenes y optimizar el desempeño de los modelos, alcanzando mejoras significativas en métricas como mAP@50 y mAP@50–95.

Los resultados obtenidos evidencian el potencial de estas herramientas para ser integradas en sistemas reales de monitoreo ambiental y vigilancia territorial, contribuyendo a la lucha contra la minería ilegal en la Amazonía colombiana. A su vez, el enfoque metodológico propuesto puede ser transferido a otros escenarios de análisis remoto, como la detección de deforestación, el monitoreo de cuerpos hídricos y la planificación de intervenciones sobre el territorio.

Se recomienda continuar con la expansión del conjunto de datos, incluyendo clases adicionales representativas de actividades mineras ilegales, así como explorar arquitecturas avanzadas que permitan una mayor precisión en entornos hostiles. La colaboración interdisciplinaria entre ingenieros, biólogos, militares y comunidades locales será clave para escalar estos desarrollos hacia aplicaciones de mayor impacto social, ambiental y geopolítico.

% TODO

\bibliographystyle{IEEEtran}
\bibliography{references}

\end{document}