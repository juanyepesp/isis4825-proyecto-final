\section{Materiales}

Se cuenta con aproximadamente 25 minutos de video etiquetados, que fueron divididos en alrededor de 4000 fotogramas en formato png, los cuales le pertenecen a la FAC, las cuales son obtenidas de cámaras FLIR (Forward-Looking Infrared), los cuales son sensores térmicos avanzados que permiten la detección de objetos y actividades a partir de radiación infrarroja emitida por los cuerpos. Los fotogramas tienen varias capturas de diferentes tipos de territorio, entre montañas, cuerpos de agua, carretera, vehículos, entre otros; cuenta también con coordenadas de imágenes para la identificación de objetos dentro de la misma y con problemas como exceso de ruido, distorsión, sombra y baja resolución de los fotogramas. Se adjunta el notebook con un primer procesamiento de las imágenes para mejorar su calidad y disminuir el ruido en las mismas. 