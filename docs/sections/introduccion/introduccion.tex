\section{Introducción}

\subsection{Objetivo General}

Diseñar y desarrollar un modelo basado en aprendizaje profundo (deep learning) para el procesamiento de imágenes extraídas de videos, para mejorar su calidad, eliminar ruido y automatizar la detección de cuerpos de agua y carreteras no pavimentadas, facilitando así la identificación automática de estos elementos y contribuyendo al reconocimiento de campamentos de minería ilegal.

\subsection{Objetivos Específicos}

\begin{enumerate}
    \item Implementar técnicas de preprocesamiento de imágenes que permitan mejorar la calidad de los fotogramas extraídos de los videos, eliminando ruido y mejorando el contraste, con una reducción mínima del 20\% en la distorsión de la imagen. 
    \item Diseñar y entrenar un conjunto de datos etiquetado para la identificación de carreteras no pavimentadas y cuerpos de agua.
    \item Desarrollar un modelo de aprendizaje profundo basado en redes neuronales convolucionales para la detección y clasificación de carreteras no pavimentadas y cuerpos de agua en videos capturados por cámaras FLIR.
    \item Evaluar el desempeño del modelo mediante métricas de clasificación como precisión, recall y F1-score, con un umbral mínimo de 80\% en cada una, para validar su aplicabilidad en escenarios reales de detección de minería ilegal.
\end{enumerate}