\section{Estado del arte}

Recientemente, el aprendizaje automático y la visión por computadora han mejorado significativamente el monitoreo ambiental mediante imágenes satelitales, especialmente en áreas como la detección de pistas clandestinas en la Amazonía. En este contexto, Pardini et al. \cite{pardini2025} desarrollaron un modelo basado en redes neuronales para reducir los falsos positivos en un 26.6\%, optimizando el procesamiento de 43 a 32 horas mediante paralelización.

En la gestión hídrica, Albuquerque Teixeira et al. \cite{albuquerque2024} aplicaron redes neuronales para segmentar embalses en Brasil, alcanzando una precisión del 95\% en la Intersección sobre Unión (IoU), mejorando así la monitorización de recursos hídricos en zonas semiáridas.

Por otro lado, Ferreira et al. \cite{ferreira2019} implementaron una fusión de datos de teledetección para detectar cultivos ilícitos, alcanzando una precisión del 92.16\% y reduciendo los falsos positivos al 5.87\%. Este enfoque es particularmente útil para identificar cultivos ilegales, como la \textit{Cannabis Sativa}, en Brasil.

Pinto Hidalgo et al. \cite{pinto2023} crearon un modelo con redes neuronales y datos geoespaciales para detectar infraestructuras de producción de coca en la frontera Venezuela-Colombia. Su metodología mejoró la vigilancia en zonas de difícil acceso, usando imágenes satelitales y bases de datos geoespaciales.

En cuanto a la detección de carreteras ilegales, Sloan et al. \cite{sloan2024} utilizaron redes U-Net y ResNet-34 para identificar carreteras en Asia Pacífico con una precisión entre el 72\% y el 81\%. Su metodología podría aplicarse en ecosistemas tropicales para monitorear el impacto ambiental de infraestructuras no autorizadas.

A pesar de estos avances, aún existen desafíos en la optimización de modelos para diferentes condiciones geográficas y climáticas, y en la reducción de costos computacionales sin afectar la precisión. Estos estudios ofrecen valiosas herramientas para mejorar la vigilancia ambiental y la gestión sostenible de los recursos naturales.
